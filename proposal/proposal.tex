\documentclass[journal]{vgtc}                % final (journal style)
%\documentclass[review,journal]{vgtc}         % review (journal style)
%\documentclass[widereview]{vgtc}             % wide-spaced review
%\documentclass[preprint,journal]{vgtc}       % preprint (journal style)

%% Uncomment one of the lines above depending on where your paper is
%% in the conference process. ``review'' and ``widereview'' are for review
%% submission, ``preprint'' is for pre-publication, and the final version
%% doesn't use a specific qualifier.

%% Please use one of the ``review'' options in combination with the
%% assigned online id (see below) ONLY if your paper uses a double blind
%% review process. Some conferences, like IEEE Vis and InfoVis, have NOT
%% in the past.

%% Please note that the use of figures other than the optional teaser is not permitted on the first page
%% of the journal version.  Figures should begin on the second page and be
%% in CMYK or Grey scale format, otherwise, colour shifting may occur
%% during the printing process.  Papers submitted with figures other than the optional teaser on the
%% first page will be refused. Also, the teaser figure should only have the
%% width of the abstract as the template enforces it.

%% These few lines make a distinction between latex and pdflatex calls and they
%% bring in essential packages for graphics and font handling.
%% Note that due to the \DeclareGraphicsExtensions{} call it is no longer necessary
%% to provide the the path and extension of a graphics file:
%% \includegraphics{diamondrule} is completely sufficient.
%%
\ifpdf%                                % if we use pdflatex
  \pdfoutput=1\relax                   % create PDFs from pdfLaTeX
  \pdfcompresslevel=9                  % PDF Compression
  \pdfoptionpdfminorversion=7          % create PDF 1.7
  \ExecuteOptions{pdftex}
  \usepackage{graphicx}                % allow us to embed graphics files
  \DeclareGraphicsExtensions{.pdf,.png,.jpg,.jpeg} % for pdflatex we expect .pdf, .png, or .jpg files
\else%                                 % else we use pure latex
  \ExecuteOptions{dvips}
  \usepackage{graphicx}                % allow us to embed graphics files
  \DeclareGraphicsExtensions{.eps}     % for pure latex we expect eps files
\fi%

%% it is recomended to use ``\autoref{sec:bla}'' instead of ``Fig.~\ref{sec:bla}''
\graphicspath{{figures/}{pictures/}{images/}{./}} % where to search for the images

\usepackage{microtype}                 % use micro-typography (slightly more compact, better to read)
\PassOptionsToPackage{warn}{textcomp}  % to address font issues with \textrightarrow
\usepackage{textcomp}                  % use better special symbols
\usepackage{mathptmx}                  % use matching math font
\usepackage{times}                     % we use Times as the main font
%\renewcommand*\ttdefault{txtt}         % a nicer typewriter font
\usepackage{cite}                      % needed to automatically sort the references
\usepackage{tabu}                      % only used for the table example
\usepackage{booktabs}                  % only used for the table example
%% We encourage the use of mathptmx for consistent usage of times font
%% throughout the proceedings. However, if you encounter conflicts
%% with other math-related packages, you may want to disable it.

%% In preprint mode you may define your own headline.
%\preprinttext{To appear in IEEE Transactions on Visualization and Computer Graphics.}

%% If you are submitting a paper to a conference for review with a double
%% blind reviewing process, please replace the value ``0'' below with your
%% OnlineID. Otherwise, you may safely leave it at ``0''.
\onlineid{0}

%% declare the category of your paper, only shown in review mode
\vgtccategory{Research}
%% please declare the paper type of your paper to help reviewers, only shown in review mode
%% choices:
%% * algorithm/technique
%% * application/design study
%% * evaluation
%% * system
%% * theory/model
\vgtcpapertype{proposal}

%% Paper title.
\title{Using Data Visualization to Pinpoint Mistakes in MIDI Practice Recordings}

%% This is how authors are specified in the journal style

%% indicate IEEE Member or Student Member in form indicated below
\author{Jeremy Grifski and Stephen Wu}
\authorfooter{
%% insert punctuation at end of each item
\item
 Jeremy Grifski is a student at The Ohio State University. E-mail: grifski.1@osu.edu.
\item
 Stephen Wu is a student at The Ohio State University. E-mail: wu.2719@osu.edu.
}

%other entries to be set up for journal
\shortauthortitle{Biv \MakeLowercase{\textit{et al.}}: Global Illumination for Fun and Profit}
%\shortauthortitle{Firstauthor \MakeLowercase{\textit{et al.}}: Paper Title}

%% Abstract section.
\abstract{
When a musician wants to practice their instrument, they often have to rely
on their peers or an instructor to help them isolate mistakes in their
technique. As an alternative solution, we are proposing a system to answer the
following question: how can we leverage data visualization to pinpoint mistakes
in music data? For the sake of scope, we have chosen to focus on MIDI recordings.
} % end of abstract

%% Keywords that describe your work. Will show as 'Index Terms' in journal
%% please capitalize first letter and insert punctuation after last keyword
\keywords{Music, Data Visualization, MIDI.}

%% ACM Computing Classification System (CCS).
%% See <http://www.acm.org/class/1998/> for details.
%% The ``\CCScat'' command takes four arguments.

\CCScatlist{ % not used in journal version
 \CCScat{K.6.1}{Management of Computing and Information Systems}%
{Project and People Management}{Life Cycle};
 \CCScat{K.7.m}{The Computing Profession}{Miscellaneous}{Ethics}
}

\vgtcinsertpkg

%%%%%%%%%%%%%%%%%%%%%%%%%%%%%%%%%%%%%%%%%%%%%%%%%%%%%%%%%%%%%%%%
%%%%%%%%%%%%%%%%%%%%%% START OF THE PAPER %%%%%%%%%%%%%%%%%%%%%%
%%%%%%%%%%%%%%%%%%%%%%%%%%%%%%%%%%%%%%%%%%%%%%%%%%%%%%%%%%%%%%%%%

\begin{document}

%% the only exception to this rule is the \firstsection command
\firstsection{Introduction}

\maketitle

Music is a profession and hobby enjoyed by many people. Unfortunately, the
field hasn't received a lot of attention from the technology community. To
this day, musicians still practice their instruments with little to no
benefit from technology.

One area of music that could really benefit from a technological upgrade would
be practice. After all, practice is usually something that occurs alone without
a lot of feedback. Without access to an instructor, musicians may find it difficult
to self-assess their abilities. They could all benefit from some sort of tool
to help pinpoint their mistakes.

In this project, we intend to build a data visualization dashboard which can
be used to compare practice MIDI files with professional MIDI files. The goal
is to isolate areas in the practice file which are most unlike the professional
file for the sake of improvement.

\section{Research Questions}

As mentioned previously, the major research question we will be looking to
address is the following: how can we leverage data visualization to pinpoint
mistakes in MIDI practice recordings?

Naturally, this question raises several underlying questions such as:

\begin{itemize}
\item How do we prioritize practice areas (pitch, tempo, etc.) in our visualization?
\item What are the most effective ways of visualizing those practice areas?
\item Is frequency data such as note frequency valuable to a musician?
\end{itemize}

In an effort to pinpoint mistakes, we will need to find the best ways to
represent our musical data, so the user will see value in the tool.

\section{Design Goals}

At a high level, we intend to construct a dashboard split into two panes: the
file pane and the graph pane.

The file pane will contain a list of active MIDI files which are each given a
color for encoding purposes. That means the dashboard will be able to support
at most 19 simultaneous MIDI files. This should be more than enough considering
the practicality of comparing that many recordings.

Each file in the file pane will be able to be selected for viewing purposes
in the graph pane. When unselected, the file's background color will be neutral.
When selected, the file's background color will mirror its color in the graph
pane.

Meanwhile, the graph pane will contain several graphs:

\begin{itemize}
\item Notes versus Time (master graph)
\item Notes versus Frequency
\item Velocity versus Time
\item Sustain versus Time
\end{itemize}

As a stretch goal, each graph will be connected with the master graph for
filtering purposes. When a section of time is selected in the master graph,
all other graphs will be updated to reflect the new subsection of data. This
will allow a user to hone in on specific mistakes.

In addition, graphs will contain tooltips which will highlight areas with
the highest amount of mistakes. These tooltips will include high level notes
to assist the user in understanding the data.

Finally, the dashboard can be extended to include realtime recording and sheet
music comparison.

\section{Hardware and Software Requirements}

To complete this project, we require the following software:

\begin{itemize}
\item JavaScript
\item D3
\item MIDI.js
\item Garage Band
\item GitHub
\item Travis CI
\end{itemize}

With this software, we should be able to build and test the entire system.

\section{Tasks and Metrics}

In order to verify the success of this project, we will be tracking several
tasks in GitHub. Namely, we will be designing and implementing the following:

\begin{itemize}
\item MIDI File Upload
\item MIDI File Pane
\item Notes versus Time Graph
\item Notes versus Frequency Graph
\item Velocity versus Time Graph
\item Sustain versus Time Graph
\item Mistake Analysis for Tooltips
\item Realtime Recording
\item Sheet Music Rendering and Comparison
\end{itemize}

Each of these tasks can probably be broken down into smaller tasks as they all
need to be designed, prototyped, and tested.

\section{Conclusion}

It may seem odd to want to think of music in a visual way, but we feel our
system will have a positive impact on musicians who want to improve their
practice sessions.

\end{document}
